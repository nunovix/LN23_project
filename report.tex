\documentclass[12pt]{article}
\usepackage[english]{babel}
\usepackage{float}
\usepackage{ragged2e}
\usepackage{indentfirst}
\usepackage{amsmath}
\usepackage{amsmath}
\usepackage{bm}
\usepackage{url}
\usepackage{booktabs}
\usepackage{hyperref}
\usepackage{float}
\usepackage{subcaption}
\usepackage{fancyhdr}
\usepackage[utf8]{inputenc}
\usepackage[margin=1.5cm, hmargin=1.2cm]{geometry}
\usepackage[utf8]{inputenc}
\usepackage{gensymb}
\usepackage{graphicx}
\usepackage[mathcal]{euscript}
\usepackage{caption}
\usepackage{wrapfig}
\usepackage{csquotes}
\usepackage[T1]{fontenc}
\usepackage{commath}
\usepackage{empheq}
\usepackage{multirow}
\usepackage{makecell}
\usepackage[maxbibnames=6, sorting=none]{biblatex}
\usepackage{multicol}
\usepackage{newpxtext,newpxmath}
\usepackage{fancyvrb}
\usepackage[dvipsnames]{xcolor}
\usepackage{eurosym}
\usepackage{comment}

\newlength\pcol
\usepackage{siunitx}

\appto{\bibsetup}{\raggedright}
\addbibresource{references.bib}

\setcounter{secnumdepth}{4}

\newcommand{\subsubsubsection}[1]{\paragraph{#1}\mbox{}\\}
\setcounter{secnumdepth}{4}
\setcounter{tocdepth}{4}


\begin{document}

%\renewcommand\thepage{}

\begin{title}

\vspace{0.5cm}

\begin{figure}[H]
\centering
  \centering
  \includegraphics[width=0.2\linewidth]{report_images/IST.png}
\end{figure}

\newcommand{\HRule}{\rule{\linewidth}{0.2mm}}
\center

\vspace{0.05cm}

\textsc{\LARGE Instituto Superior Técnico}\\[0.05cm]
\vspace{0.53cm}
\textsc{\Large  Natural Language}\\[0.2cm]

\vspace{0.1cm}

\HRule\\ 
{ \Large \bfseries Mini Project 1\\[0.02cm]}
\HRule 
\large
\center
2023/2024 

\begin{center}
\begin{minipage}{.5\linewidth}
\begin{center}
\small
\textbf{Group and relative contribution (\%)\\[0.06in]} 
Tiago Miranda, no. 93416 $\dotfill$ 50\%\\[0.01in]
Nuno Machado, no. 94021 $\dotfill$ 50\%\\

\end{center}
\end{minipage}  
\end{center}
\begin{minipage}{.9\linewidth}
\begin{center}
\small
We chose to divide the project in the following way: Nuno started solving questions a and c and Tiago questions b and d. All transducers were tested independently by both. The work wasn't performed in parallel however, both students helped each other in every question of this project.
\end{center}
\end{minipage} 



\end{title}

%\newpage


\pagestyle{fancy}
\fancyhf{}
\rhead{\textsc{\textbf{LN}}}
\lhead{\includegraphics[scale = 0.11]{images/IST.png}}
\rfoot{\thepage}

%\newpage
\thispagestyle{empty}
\renewcommand\thepage{}

\renewcommand\thepage{}
\setcounter{page}{1}
\renewcommand\thepage{\arabic{page}}
\pagestyle{fancy}

\rfoot{\thepage}

%%%%%%%%%%%%%%%%%%%%%%%%%%%%%%%%%%%%%%%%%%%%%%%%%%%%%%%%%%%%%%%%%%%%
%%%%%%%%%%%%%%%%%%%%%%%%%%%%%%%%%%%%%%%%%%%%%%%%%%%%%%%%%%%%%%%%%%%%%%%%%%%%%%
\section*{a}
\vspace{-5mm}
Since the years that will be used as inputs will always belong to the interval [2001-2099] we opted to only allow the numbers 2 and 0 for the first two digits of the given year. Also considering that the day and year must remain unchanged, if the day is inputted with a single digit, it'll be outputted with a single digit. 

\vspace{-5mm}
\section*{b}
\vspace{-5mm}
\par We opted to create an auxiliary transducer called \textbf{transpt2en.fst} that translates the month abbreviation from Portuguese to English. Then, just as in exercise a) we concatenated it with the transducer \textbf{dd\_aaaa.fst}. In order to convert a date in a mixed condensed format from English to Portuguese we inverted the \textbf{pt2en.fst} with the command \textit{fstinvert}.


\vspace{-5mm}
\section*{c}
\vspace{-5mm}
Firstly we defined the transducers \textbf{day.fst}, \textbf{month.fst} and \textbf{year.fst}. Then to make the transducer \textbf{datenum2text.fst} we created two extra transducers: \textbf{comma.fst} (to add a comma) and \textbf{slash.fst} (to consider a slash and remove it). All of these were concatenated in pairs following the order: \textbf{month.fst}, \textbf{slash.fst}, \textbf{day.fst}, \textbf{slash.fst}, \textbf{comma.fst} and \textbf{year.fst}. Since the concatenations were done in pairs several auxiliary transducers were needed. After the final one was obtained all the auxiliary ones were deleted.


\vspace{-5mm}
\section*{d}
\vspace{-5mm}
\par To obtain \textbf{mix2text.fst} we performed an union (\textit{fstunion}) of 2 transducers: the composition (\textit{fstcompose}) of \textbf{pt2en.fst}, \textbf{mix2numerical.fst} and \textbf{datenum2text.fst}; and the composition of \textbf{mix2numerical.fst} and \textbf{datenum2text.fst}. To obtain \textbf{date2text.fst} we performed an union (\textit{fstunion}) of \textbf{datenum2text.fst} and \textbf{mix2text.fst}. Again, several auxiliary transducers were generated that were deleted after obtaining the desired results.
\vfill

\nocite{*}
\addcontentsline{toc}{section}{References}
\printbibliography

\end{document}